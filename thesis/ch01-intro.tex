

\chapter{引言}
\label{chap:introduction}


\section{研究的背景及意义}
近几年,由于计算机技术的不断发展以及互联网的不断普及,用户对于计算资源和互
联网的需求日益增长。云计算被提出以满足日益增长的用户需求。但是云计算平台目
前大多是基于虚拟化技术实现的,由于虚拟机的资源是由宿主机提供的。如何调度这
些有限的资源使其做到资源的最大利用或是资源间的分配公平成为一个亟待解决的问
题。

在此背景下各个研究单位及商业公司均在其实际应用条件下提出了自己的资源调
度算法的解决方案。目前较为常见的调度算法模型为线性规划模型或是基于原有的操
作系统中进程调度的模型。但是这两种常见的模型在多机虚拟环境下均有自己的局限
性,于是各个研究单位和商业公司均在原有算法的基础上做出了改进。并且提出了基
于层次划分的调度方案来解决这些问题。使得调度平台能够做到有效的资源分配进而做
到资源的合理利用。

云计算的重要理念是提供用户服务。由于客户的需求和消费习惯都在变,所以应
当提供简单,快捷的云计算服务来满足客户需求。因此时延成为了评判云计算服务质
量的一个重要标准。目前的调度算法在平台上大多考虑资源的合理分配,较少评
估时延的因素。所以,设计一种有效的调度算法能够减少任务的平均等待时延就成为
了目前研究中一个亟待解决的问题。

\section{国内外研究现状}
目前虚拟机资源调度问题已成为目前计算机研究的一个热门领域。各个研究单位针对
不同层次,不同应用背景的资源调度设计了许多卓有成效的调度算法。

在夏威夷大学Mark Stillwella等人的工作中\cite{ref3}首先将资源分成弹性和刚性资源,并在
此基础上介绍了线性规划的模型来解决资源分配的方案,在他的的工作同时给出了解
决这个线性规划的四种方法:贪心算法,传统线性规划解决方案和Vector Packing算
法。并对这四种方案进行了对比和分析,在区分弹性资源和刚性资源的情况下,装箱
算法可以做到较好的资源利用。但是这个算法没有考虑到资源的请求的可能是实时变
化的,并且用于考虑的主要指标是任务对资源的利用情况从而对任务的时延情况考虑
不足,从这一方面来说该算法离实际应用仍有一些距离。

瑞典的于默奥大学的Johan Tordsson的工作\cite{ref1}和西班牙的马德里康普顿斯大学
的Jose Luis Lucas-Simarro的工作\cite{ref2}中提出了在不同云计算平台间部署虚拟机,
从不同时间段资源请求一般是不同的这一现象出发,结合线性规划的模型,
提出了自己的解决方案。这套方案是比较高层次的调度,可以良好的整合资源,节约成本,
但是对平台的实际利用率,时延等性能指标上没有做出有效的贡献。

在线性规划模型的基础之上圣何塞州立大学的Shailesh Sawantcite\cite{ref11}和Paolo Campegiani
\cite{ref12}提出了应用遗传算法来解决问题。从其仿真实验结果来看,遗传算法的确
做到了提升系统的CPU利用率的效果。但是由于在资源调度的实际应用背景下,遗传算
法在求解该问题的时间耗费较大,从而很难满足实际云计算平台的调度的需求。

美国劳伦斯国家实验室Ekow Otooa\cite{ref13}和麻省理工大学的
David Karger\cite{ref14}提出了用Vector Packing算法来解决多维资源的分配问题,
但由于Vector Packing问题本身是NP-C问题,所以只能用近似解来计算。
但是,由于这种算法本身的抽象是用最少的虚拟机来满足用户的需求,
故其在负载均衡和时延性能上考虑有所不足。
与此同时北京邮电大学Xin Sun在他们的工作中\cite{ref9}采用背包算法解决问题,提高了CPU的利用率。
同时该近似算法给出了$O(NlogN)$的算法,比较高效的解决了资源调度问题。

随着技术的发展,基于各个实际应用的调度算法被提出以满足实际的具体应用,
剑桥大学的P. Barham, B. Dragovic在他们的\cite{ref4}中,提出了Xen的虚拟机,
并在CPU和VCPU的层次上提出和应用了诸如BVT\cite{ref27},SEDF,Credit算法\cite{ref23},
其中Credit算法为当前Xen的实际调度算法,而莱斯大学的D. Ongaro, A. L. Cox
的工作\cite{ref5}针对Xen的I/O技术提出了进一步的优化。
这些算法可以有效的提升单个虚拟机任务执行时延,
但是并未在全局的虚拟机计算平台上有所贡献。
针对内存的虚拟机内存平衡问题,北京大学的W. Zhao\cite{ref8}提出了基于LRU思想的预测方法,
从而动态的改变虚拟机内存大小,有效的做到了内存的平衡。

随着研究的不断深入,基于公平行的调度被提出以保证Hadoop上作业的公平性,
目前Hadoop上使用的是Fair Scheduler\cite{ref24}和Capacity Scheduler\cite{ref25},
在此应用背景下,加州大学伯克利分校的A. Ghodsi的工作\cite{ref7}和B. Hindman\cite{ref6}
成功的将经济学博弈论引入到资源调度算法中来提出了DRF(Dominant Resource Fairness
)算法。DRF算法在资源分配上具有良好的公平性,从而以其为基本调度算法的OpenMesos平台
被Twitter, Conviva, UC Berkeley, and UC San Francisco等商业公司及学校使用来管理集群。
但是DRF算法在资源利用率上面有所不足,目前成为人们改进的重点。

\section{本论文的研究内容}
本论文在分析现有的线性规划模型和Credit调度算法的基础之上,
针对虚拟机资源调度中时延性能问题,
设计了BCLP算法。

总结起来说,本文的工作和内容集中体现在以下几个方面:
\begin{enumerate}
  \item 本文简要的介绍了云计算和虚拟化的背景,对现有系统的资源调度算法进行了
	分类与介绍,同时进行了详细的分析。
  \item 针对虚拟机计算平台的执行任务的平均等待时延问题,
	对资源调度算法的线性规划模型和Credit算法进行了详细分析
  \item 结合线性规划模型和Credit算法的优点,
	提出了BCLP(Based Credit Linear Programming)算法,
	来提高虚拟机计算平台的执行任务的平均等待时延。
  \item 对BCLP算法的流程,设计方案等做出了详细的阐述。
  \item 实现和仿真BCLP算法,同时与FIFO,FirstFit以及Credit算法进行对比,
	详细的分析了实验结果。
\end{enumerate}


\section{本论文的组织结构}
本文共分为五章:

第一章:介绍当今云计算的背景,虚拟化技术的现状以及各种解决虚拟机资源调度算
法的研究情况。在此基础上,简要分析了虚拟机资源分配上的解决方案的特点及其不
足。

第二章:详细的介绍了云计算和虚拟化技术的发展和现状。对目前的虚拟机资源调度
系统按照其实际调度资源的层次将其进行了分类,并对每一层次的资源调度进行了详
细的分析和介绍。

第三章:介绍了线性规划模型在调度算法上的应用以及Xen中的Credit算法对解决任务
时延的所做的优化,并结合这两种算法的优点,提出了自己的BCLP算法。与单纯
线性规划和Credit算法相比,BCLP算法在全局时延和单个任务的处理上做出了优
化。同时给出了BCLP算法的设计思路以及算法流程,并在此基础上实现了BCLP算法。

第四章:模拟仿真实验,利用C++和Matlab对BCLP算法进行了仿真并同其他算法进行对比,
并对实验结果进行了分析与评价。

第五章:总结了目前的工作内容,针对本文不足的地方提出了以后需要继续深入研究
和改进的地方。
