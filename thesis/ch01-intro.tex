

\chapter{引言}
\label{chap:introduction}


\section{研究的背景及意义}

近几年,由于计算机技术的不断发展以及互联网的不断普及,用户对于计算资源和互
联网的需求日益增长。云计算被提出以满足日益增长的用户需求。但是云计算平台目
前大多是基于虚拟化技术实现的,由于虚拟机的资源是由宿主机提供的。如何调度这
些有限的资源使其做到资源的最大利用或是资源间的分配公平成为一个亟待解决的问
题。

在此背景下各个研究单位及商业公司均在其实际应用条件下提出了自己的资源调
度算法的解决方案。目前较为常见的调度算法模型为线性规划模型或是基于原有的操
作系统中进程调度的模型。但是这两种常见的模型在多机虚拟环境下均有自己的局限
性,于是各个研究单位和商业公司均在原有算法的基础上做出了改进。并且提出了基
于层次划分的调度方案来解决这些问题。使得调度平台能够做到有效的资源分配进而做
到资源的合理利用。

云计算的重要理念是提供用户服务。由于客户的需求和消费习惯都在变,所以应
当提供简单,快捷的云计算服务来满足客户需求。因此时延成为了评判云计算服务质
量的一个重要标准。目前的调度算法在平台上大多考虑资源的合理分配,较少评
估时延的因素。所以,设计一种有效的调度算法能够减少任务的平均等待时延就成为
了目前研究中一个亟待解决的问题。

在云计算平台中,作业往往是由多个计算任务所组成的。而这些计算任务往往不是独立存在的。一个作业下的计算任务不仅要和这个
作业下的进行通信外,有的计算任务之间还存在输入和输出的数据耦合的过程。在这种
情况下,目前的资源调度均是从整体考虑资源的划分,在这个方面对往往忽略了任务
之间特性的研究工作,尤其是关联任务的上下文关系。而这个问题是提高云计算平台性能的一个关键问题。

\section{国内外研究现状}

目前虚拟机资源调度问题已成为目前计算机研究的一个热门领域。各个研究单位针对
不同层次,不同应用背景的资源调度设计了许多卓有成效的调度算法。

在夏威夷大学Mark Stillwella等人\cite{ref3}的工作中首先将资源分成弹性和刚性资源,并在
此基础上介绍了线性规划的模型来解决资源分配的方案,在他的的工作同时给出了解
决这个线性规划的四种方法:贪心算法,传统线性规划解决方案和Vector Packing算
法。并对这四种方案进行了对比和分析,在区分弹性资源和刚性资源的情况下,装箱
算法可以做到较好的资源利用。但是这个算法没有考虑到资源的请求的可能是实时变
化的,并且用于考虑的主要指标是任务对资源的利用情况从而对任务的时延情况考虑
不足,从这一方面来说该算法离实际应用仍有一些距离。

瑞典的于默奥大学的Johan Tordsson的工作\cite{ref1}和西班牙的马德里康普顿斯大学
的Jose Luis Lucas-Simarro的工作\cite{ref2}中提出了在不同云计算平台间部署虚拟机,
从不同时间段资源请求一般是不同的这一现象出发,结合线性规划的模型,
提出了自己的解决方案。这套方案是比较高层次的调度,可以良好的整合资源,节约成本,
但是对平台的实际利用率,时延等性能指标上没有做出有效的贡献。

在线性规划模型的基础之上圣何塞州立大学的Shailesh Sawant\cite{ref10}和Paolo Campegiani\cite{ref11}
提出了应用遗传算法来解决问题。从其仿真实验结果来看,遗传算法的确
做到了提升系统的CPU利用率的效果。但是由于在资源调度的实际应用背景下,遗传算
法在求解该问题的时间耗费较大,从而很难满足实际云计算平台的调度的需求。

美国劳伦斯国家实验室Ekow Otooa\cite{ref12}和麻省理工大学的
David Karger\cite{ref13}提出了用Vector Packing算法来解决多维资源的分配问题,
但由于Vector Packing问题本身是NP-C问题,所以只能用近似解来计算。
但是,由于这种算法本身的抽象是用最少的虚拟机来满足用户的需求,
故其在负载均衡和时延性能上考虑有所不足。
与此同时北京邮电大学Xin Sun\cite{ref7}在他们的工作中采用背包算法解决问题,提高了CPU的利用率。
同时该近似算法给出了$O(NlogN)$的算法,比较高效的解决了资源调度问题。

随着技术的发展,基于各个实际应用的调度算法被提出以满足实际的具体 应用,剑桥大学的 P. Barham, B. Dragovic 在他们的\cite{ref8}中,提出了 Xen 的虚拟 机,
并在 CPU 和 VCPU 的层次上提出和应用了诸如 BVT\cite{ref14},
SEDF,Credit 算法,其中 Credit 算法为当前 Xen 的实际调度算法,而莱斯大学的 D. Ongaro, A. L. Cox 的工作针对 Xen 的 I/O\cite{ref4} 技术提出了进一步的优化。
这些算法可以有 效的提升单个虚拟机任务执行时延,但是并未在全局的虚拟机计算平台上有所贡 献。针对内存的虚拟机内存平衡问题,北京大学的
 W. Zhao\cite{ref7}提出了基于 LRU 思想的预测方法,从而动态的改变虚拟机内存大小,有效的做到了内存的平衡。


随着研究的不断深入,基于公平性\cite{ref9}的调度被提出以保证云计算平台上作业的公平性,
目前Facebook和Yahoo分别开发了的是Fair Scheduler和Capacity Scheduler\cite{ref17},FairScehduler\cite{ref16}强调作业
之间的公平而Capacity Scheduler则是强调系统用户之间的公平。

在此应用背景下,加州大学伯克利分校的A. Ghodsi\cite{ref5}的工作和B. Hindman\cite{ref6}
成功的将经济学博弈论引入到资源调度算法中来提出了DRF(Dominant Resource Fairness
)算法。DRF算法在资源分配上具有良好的公平性,从而以其为基本调度算法的Mesos平台
被Twitter, Conviva, UC Berkeley, and UC San Francisco等商业公司及学校使用来管理集群。
但是DRF算法在资源利用率上面有所不足,目前成为人们改进的重点。

在DRF算法的基础之上,加州大学伯克利分校的A.Ghodsi等人以原先的DRF
为基础,为了解决不同组织架构层级之间资源分配的公平性的问题。在原先DRF算法
的基础上,提出了分层公平的HDRF\cite{ref23}算法,这个算法很好的解决了层级分配公平的问题。
但是同时抛出了一个问题,即资源分配的情况应当结合任务和资源的位置加以分析。

在上述算法的基础之上,公司和研究机构也设计的不同的云计算平台和调度器的设计。
调度器也从传统的单一的调度器迈向了两级调度器,Apache Mesos时加州大学伯克利
分校开发的两层调度器,旨在能够在同一个计算平台上能够通用的计算运行多个不同的
计算框架任务。于此类似的工业界的实现则是已YARN为主的多框架计算平台。

在计算框架的方面,Spark,作为加州大学伯克利分校开发的类Hadoop\cite{ref29}\cite{ref31}\cite{ref32} MapReduce的通用的并行计算框架,
对于传统的MapReduce作业进行了优化,Spark将是任务间输出结果保存在内存中,减少对HDFS的读写次数,
使得Spark能更好地适用于需要多次迭代的map reduce的算法。

与此同时,Apache Tez\cite{ref30}作为YARN\cite{ref22}平台上解决DAG作业\cite{ref21}优化的计算框架。为了解决YARN平台上作业启用一个独立的ApplicationMaster
造成作业延迟较大和资源无法重用导致的作用利用率低下的问题。在这个方面采用了ApplicationMaster缓冲和Contianer重用和
预启动等策略减少作业的延时。充分提高集群的利用率。

同时从整个云计算调度器的发展来看,调度器的发展经历了从长时间的基于全局的最优化调度,和短决策的局部中心调度\cite{ref25}\cite{ref26}\cite{ref27}\cite{ref28}\cite{ref36},
再到两层的多样的调度器,最后是弱化的中心调度器。目前出现了基于短作业调度的去中心调度的Sparrow\cite{ref20}调度等。从这里
可以看出,资源调度的分配和任务调度执行特点的的层级结合时越来越紧密。

\section{本论文的研究内容}

本论文在分析现有的DRF算法的基础之上,针对有通信的任务,在DAG任务的基础之上,为了能够
提高对任务执行的效率,设计了CBDRF调度算法。

总结起来说,本文的工作和内容集中体现在以下几个方面:
\begin{enumerate}
  \item 本文简要的介绍了云计算和虚拟化的背景,对现有系统的资源调度算法进行了
	分类与介绍,同时进行了详细的分析。
  \item 针对DAG任务和相关通信关联任务,
	对资源调度算法的DRF算法和图论的相关算法进行了详细的介绍。
  \item 结合DRF算法和相关图论模型的优点,为了解决原有对通信关联任务调度造成的资源浪费,性能不足的问题
	提出了CBDRF法,来提高云计算平台资源利用率和执行性能。
  \item 对CBDRF算法的流程,设计方案等做出了详细的阐述。
  \item 实现和仿真CBDRF算法,同时与Openstack的调度算法以及其他DRF算法进行对比,
	详细的分析了实验结果。
\end{enumerate}


\section{本论文的组织结构}
本文共分为六章:

第一章:介绍当今云计算的背景,虚拟化技术的现状以及各种解决云计算资源调度算
法的研究情况。在此基础上,简要分析了云计算资源分配上的解决方案的特点及其不
足。

第二章:详细的介绍了云计算集群管理和调度。对目前的的云计算调度和管理的调度进行
了详细的分析,分析了现有的调度系统的调度模型,调度范型以及常见的资源的调度算法。

第三章:给出了CBDRF的理论基础,图论基础和公平基础,详细的介绍了CBDRF实现的
基础图论算法tarjan,拓扑排序算法和并查级。同时介绍了基于公平性的基础max-min算法
和公平的度量指标简氏公平指数。

第四章:在DAG任务基础之上,介绍了存在关联任务,结合DRF的算法和图论,提出了自己的CBDRF算法。与先前
的算法相比,CBDRF能保障全局的数据一致性,在全局时延和单个任务的处理上做出了优
化。同时给出了CBDRF算法的设计思路以及算法流程,在实现CBDRF算法的同时,给出了基于
CBDRF算法的调度系统的实现。

第五章:模拟仿真实验,利用C++和Python对CBDRF算法进行了仿真并同其他算法进行对比,
并对实验结果进行了分析与评价。

第六章:总结了目前的工作内容,针对本文不足的地方提出了以后需要继续深入研究
和改进的地方。
