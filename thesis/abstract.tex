%% ----------------------------------------------------------------------
%% START OF FILE
%% ----------------------------------------------------------------------
%% 
%% Filename: cover.tex
%% Author: Fred Qi
%% Created: 2008-06-25 11:34:44(+0800)
%% 
%% ----------------------------------------------------------------------
%%% CHANGE LOG
%% ----------------------------------------------------------------------
%% Last-Updated: 2012-12-11 20:43:27(+0800) [by Fred Qi]
%%     Update #: 32
%% ----------------------------------------------------------------------

\begin{cabstract}
	
	云计算平台可以使用户通过互联网获得海量的计算资源。用户可以在云计算平台上进行各类计算任务和查询处理任务。
	而云计算平台的调度器作为云计算平台的核心模块,其作为变得越来越重要,面对的请求也越来越复杂。随着计算任务
	的负载增大,云计算平台对虚拟资源的管理成为云计算平台面临的新的挑战。目前的云计算平台对于存在关联关系任务
	的调度存在着资源利用率低,分配不公平导致负载不均衡,进而影响云计算平台性能的问题。本文重点研究了云计算平台
	对于存在关联性关系任务的资源调度问题,对提高云计算平台的资源利用率,资源分配公平性和负载均衡提出了 CBDRF算法
	来解决的云计算协同计算平台下的虚拟机资源调度问题。主要的研究工作有以下几个方面:
	
	(1)针对云平台计算管理资源问题,结合云计算平台的结构,对云计算平台的资源调度器和调度模型和资源分配加以分析。

	(2)针对云计算平台在资源管理对相关联的任务分配公平性不足,资源利用率低和性能不足的问题。本文利用图论的相关模型,
结合最小最大原则以及DRF算法的四习总,对云计算平台的资源调度问题进行建模。在此模型基础之上,
设计和实现了一种基于DRF和图论算法的的CBDRF调度算法。能够在任务关联公平性和执行效率之中找到平衡点,同时,
有效的提高了资源的利用情况。

	(3)在云计算平台算法的基础之上,结合云计算调度的相关研究现状和相关需求。完成了云计算调度器的设计和实现,最后
在openstack平台上进行了实际的测试和验证。

\end{cabstract}


\begin{ckeywords}
云计算,~~资源调度,~~负载均衡,~~虚拟机放置,~~资源公平性,~~关联任务~~
\end{ckeywords}
\begin{cthesistype}
应用基础研究类
\end{cthesistype}




\begin{eabstract}
\noindent{
ICloud computing platform that lets users access to vast amounts of computing resources over the Internet. On cloud computing platforms, users can conduct various computational tasks, and query processing tasks.
And scheduling as a platform for cloud computing cloud computing platform is the core module, as it becomes more and more important, in the face of requests are getting more sophisticated. As computing tasks
Load increases, cloud computing platform for virtual resource management becomes a new challenge for cloud computing platform. Relationship with the present cloud computing platform for tasks
Scheduling there are low rates of resource utilisation, inequitable distribution led to uneven loading, thereby affecting the performance of cloud computing platforms. This article focuses on cloud computing platforms
To correlation tasks with resource scheduling, cloud computing platform to improve the utilization rate of resources and equity of resource allocation and load balancing CBDRF algorithm is proposed
To address cloud computing virtual machine resource scheduling problems in collaborative computing platforms. Major research efforts in the following areas:}\\

\noindent{(1) management for cloud computing resources, combined with the structure of the cloud computing platform, cloud computing platform and Resource Scheduler scheduling model and resource allocation were analyzed.}\\

\noindent{(2) in resource management for cloud computing platform for the associated lack of task allocation fairness, low resource utilization and performance issues. Using graph theory model
Combined with the min-max principle and algorithm of DRF's four total acquisition, to model the resource scheduling problems with cloud computing platforms. This model is based on
Design and implement a DRF CBDRF scheduling algorithm and graph algorithms. Able to associate tasks with balance of fairness and efficiency at the same time,
Effectively improve the utilization of resources.}\\

\noindent{(3) in the cloud algorithm based on combination of scheduling and the research status of cloud computing and related needs. Completed the design and implementation of cloud computing Scheduler, and finally
The openstack platform has been tested and validated.}
\end{eabstract}


\begin{ekeywords}
Cloud Computing, Resource Scheduling, Load Balancing, Resource Fairness, Correlation Task
\end{ekeywords}

\begin{ethesistype}
Algorithm Desigin
\end{ethesistype}

% \tableofcontents

%% ----------------------------------------------------------------------
%%% END OF FILE 
%% ----------------------------------------------------------------------