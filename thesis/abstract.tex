%% ----------------------------------------------------------------------
%% START OF FILE
%% ----------------------------------------------------------------------
%% 
%% Filename: cover.tex
%% Author: Fred Qi
%% Created: 2008-06-25 11:34:44(+0800)
%% 
%% ----------------------------------------------------------------------
%%% CHANGE LOG
%% ----------------------------------------------------------------------
%% Last-Updated: 2012-12-11 20:43:27(+0800) [by Fred Qi]
%%     Update #: 32
%% ----------------------------------------------------------------------

\begin{cabstract}
	
	在互联网时代,云计算是一种能将大量的计算资源整合,使得用户能够按照需要运行大量的计算任务的新兴计算模式。
	云计算平台的调度器作为资源整合的核心模块,在优化资源,提高计算任务性能以及降低平台风险上起着重要的作用。
	随着云计算平台上的任务运行的任务更加复杂,更加多样化,云计算平台的调度器的作用变得越来越重要。随着计算任务
	的负载增大,云计算平台对虚拟资源的管理成为云计算平台面临的新的挑战。目前的云计算平台对于存在关联关系任务
	的调度存在着资源利用率低,分配不公平导致负载不均衡,进而影响云计算平台性能的问题。
	
	针对如何提高云计算平台资源利用率、资源分配公平性,以及负载均衡问题,提出了面向关联关系的主导资源公平性算法(CBDRF)。
	主要的研究工作有以下几个方面:
	
	(1)针对云平台计算管理资源问题,结合云计算平台的结构,重点分析了资源分配的特点、平台调度模型以及平台调度器。

	(2)针对云计算平台中关联任务资源分配时公平性不足、资源利用率低等问题。本文采用图论的相关模型,
结合最小最大原则以及DRF算法,对云计算平台的资源调度问题进行建模。在此模型基础之上,
设计实现了一种基于DRF和图论算法的的CBDRF调度算法。能够在任务关联公平性和执行效率之中平衡,
有效的提高了资源的利用情况。

	(3)在实现CBDRF算法的基础之上,结合云计算调度的研究现状和需求。完成了云计算调度器的设计实现并
在Openstack平台上进行了实际的测试验证。

\end{cabstract}
\begin{ckeywords}
云计算,~资源调度,~负载均衡,~虚拟机放置,~资源公平性,~关联任务~~
\end{ckeywords}
\begin{cthesistype}
应用基础研究类
\end{cthesistype}




\begin{eabstract}
\noindent{
In the age of the Internet, cloud computing is a way to turn a large amount of computing resources as needed which enables users to run a large number of emerging model of computing tasks.
Scheduler of the cloud computing platform which is the core of integration modules plays an important role in optimizing resources, improving task performance and reducing risk platform.
As the task  which is worked on the cloud computing platform becomes more and more complex, more and more diverse, the role of the scheduler becomes more and more important in cloud computing platform. 
As computing tasks Load increases, cloud computing platform for virtual resource management becomes a new challenge for cloud computing platform. Relationship with the present cloud computing platform for tasks
Scheduling there are low rates of resource utilisation, inequitable distribution led to uneven loading, thereby affecting the performance of cloud computing platforms.}\\

\noindent{In Order to Improve resource utilization, resource allocation fairness cloud computing platforms, as well as load balancing problems, we presents correlationship-based-domainant-fairness algorithm (CBDRF).
Major research efforts in the following areas.}\\

\noindent{(1) In the research of management for cloud computing resources, combined with the structure of the cloud computing platform, focuses on the characteristics of resource allocation, scheduling model platform, and the platform Scheduler.}\\

\noindent{(2) To Solve the problem of the correlationship tasks in the allocation of resources for cloud computing platform of low resource utilization and insufficient equity. We use graph theory model
Combined algorithm for min-max principle and the DRF, to model the resource scheduling problems with cloud computing platforms. Based on this Model  we 
design and implement CBDRF scheduling algorithm which based on DRF algorithm and graph algorithms which is able to schedule the correlationship tasks with balance of fairness and efficiency,
and we effectively improve the utilization of resources.}\\

\noindent{(3) Based on the realization of CBDRF algorithm and combining cloud computing schedule research present situation and needs. we complete implementation of scheduler and
tests the algorithm on the Openstack platform.}
\end{eabstract}


\begin{ekeywords}
Cloud Computing, Resource Scheduling, Load Balancing, Resource Fairness, Correlation Task
\end{ekeywords}

\begin{ethesistype}
Applied Basic Research
\end{ethesistype}

% \tableofcontents

%% ----------------------------------------------------------------------
%%% END OF FILE 
%% ----------------------------------------------------------------------