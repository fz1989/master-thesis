%% ----------------------------------------------------------------------
%% START OF FILE
%% ----------------------------------------------------------------------
%% 
%% Filename: ch03-frontmatter.tex
%% Author: Fred Qi
%% Created: 2012-12-26 09:12:47(+0800)
%% 
%% ----------------------------------------------------------------------
%%% CHANGE LOG
%% ----------------------------------------------------------------------
%% Last-Updated: 2013-03-12 08:16:15(+0800) [by Fred Qi]
%%     Update #: 19
%% ----------------------------------------------------------------------
\chapter{CBDRF算法的相关技术}
\label{chap:outline}
\section{图的联通性算法}
\subsection{强联通分量}
在有向图G中,如果两个顶点可以相互通达,则称两个顶点强连通(strongly connected)。
如果有向图G的每两个顶点都强连通,称G是一个强连通图。
非强连通图有向图的极大强连通子图,称为强连通分量(strongly connected components)。

下图中,子图{1,2,3}为一个强连通分量,因为顶点1,2,3两两可达。{4},{5,6}也分别是两个强连通分量。

直接根据定义,用双向遍历取交集的方法求强连通分量,时间复杂度为O(N*N+M)。更好的方法有算法有,Tarjan算法,Kosaraju算法。
其中Tarjan算法和Kosaraju算法均是是基于对图深度优先搜索的算法,与Kosaraju算法不同,
Tarjan算法只进行一遍深度优先搜索索,较Kosaraju算法进行两遍深度优先搜索有30\%的性能提升。

Tarjan算法以一个有向图G作为输入,每个强连通分量为搜索树中的一棵子树
首先定义结点u的深度优先搜索标号DFN(u),表示节点u是被访问的次序编号。
此外,每个结点u还有一个值Low(u),表示从u出发经有向边可到达的所有结点中最小的次序号。
显然Low(u)总是不大于DFN(u),且当从v出发经有向边不能到达其他结点时,这两个值相等,
此时,以u为根的搜索子树上的所有节点属于同一个强连通分量。
其中Low(u)在深度优先搜索的过程中求得,通过上述可以发现以u为根的搜索子树上的所有节点属于同一个
强连通分量当且仅当DFN(u)=Low(u)时。而Low(u)的计算由以下公式给出:
$$Low(u)=\min
\begin{cases}
DFN(u) & \\
Low(v)& \text{(u,v)为树枝边,u为v的父节点}\\
DFN(v)& \text{(u,v)为指向栈中节点的后向边(非横叉边)}
\end{cases}$$

由此可得,Tarjan算法的基本流程如下:
\begin{enumerate}
\item 任选图G中的未被访问的结点u开始进行深度优先搜索遍历,如果深度优先搜索结束后仍有未访问的结点,则再从中任选一点再次进行。
\item 搜索过程中标记已访问的节点,如果是已访问的结点不再访问。
\item 搜索时,把当前搜索树中未处理的节点加入一个堆栈,回溯时可以判断栈顶到栈中的节点是否为一个强连通分量。
\end{enumerate}

由此可以得到Tarjan算法在搜索时的主要的伪代码
\begin{algorithm} 
\caption {Tarjan Algorithm} 
\begin{codebox}
\Procname{$\proc{tarjan}(u)$}
\li	$Index \leftarrow time + 1$
\li	$DFN[u] \leftarrow time$
\li	$Low[u] \leftarrow time$
\li	$Stack.push(u)$
\li	\For $each (u, v)$ $in$ $E[u]$
\li		\Do  \If $\proc{visted}(v)$
\li			\Then
				$\proc{tarjan}(v)$                  
\li            			$Low[u] \leftarrow \min(Low[u], Low[v])$
\li        		 \ElseIf $\proc{inStack}(v)$ 
\li            			\Then $Low[u] \leftarrow \min(Low[u], DFN[v])$
			\End
		\End
\li	\If $DFN[u] \isequal Low[u]$
\li		\Then \While $u != v$
\li				\Do $v \leftarrow$ $\proc{Stack.pop}()$                  
\li            				$\proc{print}(v)$
				\End 
		\End
\end{codebox}
\end{algorithm} 
\subsection{拓扑排序}
在图论中,如果一个有向图无法从某个顶点出发经过若干条边回到该点,则这个图是一个有向无环图(DAG图)。
因为有向图中一个点经过两种路线到达另一个点未必形成环,
因此有向无环图未必能转化成树,但任何有向树均为有向无环图。如右图,不为有向树,但为有向无环图。

对一个有向无环图(Directed Acyclic Graph简称DAG)G进行拓扑排序,是将G中所有顶点排成一个线性序列,
使得图中任意一对顶点u和v,若边(u,v)∈E(G),则u在线性序列中出现在v之前。
通常,这样的线性序列称为满足拓扑次序(Topological Order)的序列,
简称拓扑序列。简单的说,由某个集合上的一个偏序得到该集合上的一个全序,这个操作称之为拓扑排序。

一个较大的工程往往被划分成许多子工程,我们把这些子工程称作活动(activity)。
在整个工程中,有些子工程(活动)必须在其它有关子工程完成之后才能开始,
也就是说,一个子工程的开始是以它的所有前序子工程的结束为先决条件的,
但有些子工程没有先决条件,可以安排在任何时间开始。
为了形象地反映出整个工程中各个子工程(活动)之间的先后关系,
可用一个有向图来表示,图中的顶点代表活动(子工程),图中的有向边代表活动的先后关系
,即有向边的起点的活动是终点活动的前序活动,只有当起点活动完成之后,其终点活动才能进行。
通常,我们把这种顶点表示活动、边表示活动间先后关系的有向图称做顶点活动网(Activity On Vertex network),
简称AOV网。

在AOV网中,若不存在回路,则所有活动可排列成一个线性序列,使得每个活动的所有前驱活动都排在该活动的前面,我们把此序列叫做拓扑序列(Topological order),由AOV网构造拓扑序列的过程叫做拓扑排序(Topological sort)。AOV网的拓扑序列不是唯一的,满足上述定义的任一线性序列都称作它的拓扑序列。

由AOV网构造拓扑序列的拓扑排序算法主要是循环执行以下两步,直到不存在入度为0的顶点为止。
\begin{algorithm} 
\caption {Topological Sort} 
\begin{codebox}
\Procname{$\proc{TopologicalSort}(G)$}
\li Queue.clear()
\li \For $each$ $vertex $ $u$ $in$ $G$
\li	\Do \If $\attribii{u}{indeg}  \isequal 0$
\li		\Then	$Q.push(u)$	
\li				$isVisited[u] \leftarrow True$
		\End
	\End
\li \While $\attribii{Queue}{\proc{size}()} > 0$
\li	\Do	$u \gets \attribii{Queue}{\proc{front}()}$
\li		$\attribii{Queue}{\proc{pop}()}$
\li		\For $each$ $(u,v)$ $in$ $ Edge[v]$
\li			\Do	$ \attribii{v}{indeg} \leftarrow  \attribii{v}{indeg} - 1$
\li			\If $\attribii{v}{indeg}  \isequal $0 and $isVisited[v] \isequal False$
\li				\Then	$Q.push(v)$
\li						$isVisited[v] \leftarrow True$
				\End
	\End
   \End
\end{codebox}
\end{algorithm} 

\begin{enumerate}
\item 选择一个入度为0的顶点并输出之;
\item 从网中删除此顶点及所有出边。
\item 循环结束后,若输出的顶点数小于网中的顶点数,则输出“有回路”信息,否则输出的顶点序列就是一种拓扑序列。
\end{enumerate}

\subsection{并查集}
在计算机科学中,并查集是一种树型的数据结构,其保持着用于处理一些不相交集合(Disjoint Sets)的合并及查询问题。有一个联合-查找算法(union-find algorithm)定义了两个操作用于此数据结构:

\begin{itemize}
\item Find:确定元素属于哪一个子集。它可以被用来确定两个元素是否属于同一子集。
\item Union:将两个子集合并成同一个集合。
\end{itemize}

因为它支持这两种操作,一个不相交集也常被称为联合-查找数据结构(union-find data structure)或合并-查找集合(merge-find set)。
其他的重要方法,MakeSet,用于建立单元素集合。有了这些方法,许多经典的划分问题可以被解决。

为了更加精确的定义这些方法,需要定义如何表示集合。一种常用的策略是为每个集合选定一个固定的元素,称为代表,以表示整个集合。接着。Find(x)返回x所属集合的代表,而Union使用两个集合的代表作为参数。


并查集实现方式有多种,包括并查集链表和并查集森林,其中并查集森林是并查集的高效实现。并查集森林是一种将每一个集合以树表示的数据结构,其中每一个节点保存着到它的父节点的引用。这个数据结构最早由Bernard A. Galler和Michael J. Fischer于1964年提出,但是经过了数年才完成了精确的分析。

在并查集森林中,每个集合的代表即是集合的根节点。“查找”根据其父节点的引用向根行进直到到底树根。“联合”将两棵树合并到一起,这通过将一棵树的根连接到另一棵树的根。实现这样操作的一种方法是:

这是并查集森林的最基础的表示方法,这个方法不会比链表法好,这是因为创建的树可能会严重不平衡;然而,可以用两种办法优化。

第一种优化方式是按秩合并,即总是将更小的树连接至更大的树上。因为影响运行时间的是树的深度,
更小的树添加到更深的树的根上将不会增加秩除非它们的秩相同。在这个算法中,术语“秩
”替代了“深度”,因为同时应用了路径压缩时(见下文)秩将不会与高度相同。
单元素的树的秩定义为0,当两棵秩同为r的树联合时,它们的秩r+1。
只使用这个方法将使最坏的运行时间提高至每个MakeSet、
Union或Find操作O($\log n$)。采用按秩合并的Union伪代码如下所示:
\begin{algorithm} 
	\caption {Union-Set Union} 
	\begin{codebox}
		\Procname{$\proc{Union}(u,v)$}
\li			$rootu \leftarrow \proc{Find}(u)$
\li			$rootv \leftarrow \proc{Find}(v)$
\li			\If	$rootu$ $!=$ $rootv$
\li				\Then
					\If $\attribii{rootu}{rank} < \attribii{rootv}{rank}$
\li						\Then
							$\attribii{rootu}{parent} \leftarrow \attribii{rootv}{parent}$
\li							$\attribii{rootv}{rank} \leftarrow \attribii{rootv}{rank} + \attribii{rootu}{rank}$
\li						\Else
\li							$\attribii{rootv}{parent} \leftarrow \attribii{rootu}{parent}$
\li							$\attribii{rootu}{rank} \leftarrow  \attribii{rootv}{rank} + \attribii{rootu}{rank}$
						\End
				\End
\li			\Return	$\attribii{u}{parent}$
	\end{codebox}
\end{algorithm} 

另外一种优化的方式是采用路径压缩,这是一种在执行“查找”时扁平化树结构的方法。
关键在于在路径上的每个节点都可以直接连接到根上;
他们都有同样的表示方法。
为了达到这样的效果,Find递归地经过树,
改变每一个节点的引用到根节点。
得到的树将更加扁平,
为以后直接或者间接引用节点的操作加速。
采用路径压缩的Find伪代码如下所示:
\begin{algorithm} 
	\caption {Union-Set Find} 
	\begin{codebox}
		\Procname{$\proc{Find}(u)$}
\li			\If $\attribii{u}{parent}$ $!=$ $root$
\li			\Then	$\attribii{u}{parent} \leftarrow \proc{Find}(\attribii{u}{parent})$ \End
\li			\Return	$\attribii{u}{parent}$
	\end{codebox}
\end{algorithm} 

这两种技术可以互补,可以应用到另一个上,每个操作的平均时间仅为O($\alpha(n)$),$\alpha(n)$是n = f(x) = A(x,x)的反函数,并且A是急速增加的阿克曼函数。因为$\alpha$(n)是其的反函数,$\alpha$(n)对于可观的巨大n还是小于5。因此,平均运行时间是一个极小的常数。

\section{资源分配公平性}
\subsection{max-min公平}
我们经常面临给一组用户划分稀有资源的问题,他们都享有等价的权利来获取资源,
一种在实际中广泛使用的分享技术称作“最大最小公平分享”.直观上,公平分享分配给
每个用户想要的可以满足的最小需求,然后将没有使用的资源均匀的分配给需要‘大资源’的用户。

最大最小公平分配算法的形式化定义如下:
资源按照需求递增的顺序进行分配
不存在用户得到的资源超过自己的需求
未得到满足的用户等价的分享资源
与之对应的可执行定义:

考虑用户集合1, …, n分别有资源需求x1, x2, …, xn.不失一般性,
令资源需求满足x1 <= x2 <= … <= xn.令服务器具有能力C.
那么,我们初始把C/n资源给需求最小的用户.这可能会超过用户1的需求,继续处理.该过程结束时,
每个用户得到的没有比自己要求更多,而且,如果其需求得不到满足,
得到的资源也不会比其他用户得到的最多的资源还少.我们之所以称之为最大最小公平分配是因为我们最大化了资源得不到满足的用户最小分配的资源.

常见的max-min的公平性主要有两种,一种是不带权公平性分配,另一种则是带权公平性分配。

例如:有一四个用户的集合,资源需求分别是2,2.6,4,5,其资源总能力为10,其max-min的最小计算方式如下所示:

第一轮,我们暂时将资源划分成4个大小为2.5的.由于这超过了用户1的需求,这使得剩了0.5个均匀的分配给剩下的3个人资源,给予他们每个2.66.这又超过了用户2的需求,所以我们拥有额外的0.066
来分配给剩下的两个用户,给予每个用户2.5+0.66…+0.033…=2.7.因此公平分配是:用户1得到2,用户2得到2.6,用户3和用户4每个都得到2.7.

不带权最大最小公平假设所有的用户拥有相同的权利来获取资源.有时候我们需要给予一些用户更大的配额.特别的,尤其是有些用户具有一定的优先级时候
我们可能会给不同的用户1, …, n关联权重w1, w2, …, wn,这反映了他们间的资源配额.

我们通过定义带权的最大最小公平分配来扩展最大最小公平分配的概念以使其包含这样的权重:

例如:资源需求分别是4,2,10,4,权重分别是2.5,4,0.5,1,资源总能力是16,其分配的方法如下所示:

第一步是标准化权重,将最小的权重设置为1.这样权重集合更新为5,8,1,2.这样我们就假装需要的资源不是4份而是5+8+1+2=16份.因此将资源划分成16份.在资源分配的每一轮,我们按照权重的比例来划分资源,因此,在第一轮,我们计算C/n为16/16=1.在这一轮,用户分别获得5,8,1,2单元的资源,用户1得到了5个资源,但是只需要4,所以多了1个资源,同样的,用户2多了6个资源.用户3和用户4拖欠了,因为他们的配额低于需求.现在我们有7个单元的资源可以分配给用户3和用户4.他们的权重分别是1和2,最小的权重是1,因此不需要对权重进行标准化.给予用户3额外的7 × 1/3单元资源和用户4额外的7 × 2/3单元.这会导致用户4的配额达到了2 + 7 × 2/3 = 6.666,超过了需求.所以我们将额外的2.666单元给用户3,最终获得1 + 7/3 + 2.666 = 6单元.最终的分配是,4,2,6,4,这就是带权的最大最小公平分配.
\subsection{公平性的度量}
为评价分配算法的公平性,需要对算法得到的分配结果进行定量的度量,通过度量值判定资源分配算法在该应用背景下的公平性,为资源分配算法的选择提供定量的评估标准。
常见的公平性的度量函数有简氏指数,这是一种对单资源定义公平性的评价,如果分配的向量$X({x}_{1},{x}_{2},...,{x}_{n})$如下所示:
\begin{equation}
Jain(\textbf{X}) = \frac{{(\sum_{i=1}^{n}{x}_{i})}^{2}}{n\sum_{i=1}^{n}{{x}_{i}}^{2}}
\end{equation}
分析公式可以看出,公式的取值时1/n到1的,进一步分析发现
当资源分配最为公平的时候由于各项相等,其简式系数计算的结果为1。
而当资源分配最为不公平的时候,即存在某个独占的情况,其计算结果为1/n。
在此基础之上有一个基于系数$\beta$的单资源统一分配函数,在此基础之上,可以调整$\beta$的取值来采用不同的评价函数进行参数调优。
其公式的定义如下所示:
\begin{equation}
{f}_{\beta} (\textbf{X})= sgn(1-\beta)*{\left[\sum_{i=1}^{n}{\left(\frac{{x}_{i}}{\sum_{j=1}^{n}{x}_{j}} \right)}^{1 - \beta} \right]}^{\frac{1}{\beta}}
\end{equation}




其中$X({x}_{1},{x}_{2},\dot,{x}_{n})$仍为其中的分配结果向量。


\section{本章小结}
%% ----------------------------------------------------------------------
%%% END OF FILE 
%% ----------------------------------------------------------------------
