%% ----------------------------------------------------------------------
%% START OF FILE
%% ----------------------------------------------------------------------
%% 
%% Filename: ch05-backmatter.tex
%% Author: Fred Qi
%% Created: 2012-12-26 13:08:53(+0800)
%% 
%% ----------------------------------------------------------------------
%%% CHANGE LOG
%% ----------------------------------------------------------------------
%% Last-Updated: 2012-12-26 13:10:54(+0800) [by Fred Qi]
%%     Update #: 9
%% ----------------------------------------------------------------------


\chapter{实验与结果分析}
\label{cha:backmatter}
\section{实验环境}
\subsection{实验硬件平台}
本实验的硬件环境由以下的工作站和pc节点组成的openstack平台。其中由8个工作站构成的计算节点和4台pc机组成的网络和控制节点。

其中工作站的配置如下所示:
\begin{table}[htp]
\begin{center}
\begin{tabular}{|c|c|}
\hline
CPU & Intel\textregistered Xeon E5-2620 2.0GHZ\\
\hline
核数 & 24cores\\
\hline
内存 & 128GB\\
\hline
硬盘 &  2TB\\
\hline
\end{tabular}
\end{center}
\caption{计算节点的配置}
\label{default}
\end{table}

网络和控制节点的配置如下表所示
\begin{table}[htp]
\begin{center}
\begin{tabular}{|c|c|}
\hline
CPU & Intel\textregistered Core TM i5-3470 3.2GHZ\\
\hline
核数 & 4cores\\
\hline
内存 & 4GB\\
\hline
硬盘 &  500GB\\
\hline
\end{tabular}
\end{center}
\caption{网络节点和控制节点配置}
\label{default}
\end{table}

\subsection{实验软件}
CBDRF调度算法和调度器采用了python的实现,所用的python的第三方库列举如下:
\begin{itemize}
\item Python-Tornado 

Tornado 是 FriendFeed 使用的可扩展的非阻塞式 web 服务器及其相关工具的开源版本。
主要用来进行进程间接受消息的WebAPI的通信基础。
\item Python-Requests 

这是一个Http请求的第三方库,用于进行进程间http通信。
\item Python-MySQLDB  

python连接数据库的第三方库。
\item Python-SQLAlchemy 

这是资源管理器连接资源监控的数据库的底层。
\item Python-Numpy 

Scipy的底层数学运算的实现,NumPy系统是Python的一种开源的数字扩展。这种工具可用来存储和处理大型矩阵
\item Python-Scipy 

Python的科学计算类库。Python scipy的科学计算应用。包括了数据处理,数值计算等内容,数字信号处理,快速傅立叶变换等应用。
\end{itemize}


\section{实验结果和分析}
\subsection{资源公平性测试}
3Graph,
Graph1分配情况柱状图
Graph2对比filterscheduler,,对比方面针对domainantshare和分配情况,
Graph3计算各个分配的jains 公平系数。
\subsection{资源利用率测试}
3Graph 
Graph1散点图查看DAG任务资源分配情况,对比filter和DRF的分配散点图
Graph2另外测试可分配次数,对比filter和DRF的分配次数。
Graph3同时计算测试资源实际利用情况。

\subsection{任务执行的性能测试}
3Graph
Graph1单个测试,说明关联合并的必要性
Graph2关联任务的执行情况。
Graph3调度测试对比


%% ----------------------------------------------------------------------
%%% END OF FILE 
%% ----------------------------------------------------------------------
