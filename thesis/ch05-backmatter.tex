%% ----------------------------------------------------------------------
%% START OF FILE
%% ----------------------------------------------------------------------
%% 
%% Filename: ch05-backmatter.tex
%% Author: Fred Qi
%% Created: 2012-12-26 13:08:53(+0800)
%% 
%% ----------------------------------------------------------------------
%%% CHANGE LOG
%% ----------------------------------------------------------------------
%% Last-Updated: 2012-12-26 13:10:54(+0800) [by Fred Qi]
%%     Update #: 9
%% ----------------------------------------------------------------------


\chapter{实验与结果分析}
\label{cha:backmatter}
\section{实验环境}
\subsection{实验硬件平台}
本实验的硬件环境由以下的工作站和pc节点组成的openstack平台。其中由8个工作站构成的计算节点和4台pc机组成的网络和控制节点。

其中工作站的配置如下所示:
\begin{table}[htp]
\begin{center}
\caption{计算节点的配置}
\begin{tabular}{|c|c|}
\hline
CPU & Intel\textregistered Xeon E5-2620 2.0GHZ\\
\hline
核数 & 24cores\\
\hline
内存 & 128GB\\
\hline
硬盘 &  2TB\\
\hline
\end{tabular}
\end{center}
\label{default}
\end{table}

网络和控制节点的配置如下表所示
\begin{table}[htp]
\begin{center}
\caption{网络节点和控制节点配置}
\begin{tabular}{|c|c|}
\hline
CPU & Intel\textregistered Core TM i5-3470 3.2GHZ\\
\hline
核数 & 4cores\\
\hline
内存 & 4GB\\
\hline
硬盘 &  500GB\\
\hline
\end{tabular}
\end{center}
\label{default}
\end{table}

\subsection{实验软件}
CBDRF调度算法和调度器采用了python的实现,所用的python的第三方库列举如下:
\begin{itemize}
\item Python-Tornado 

Tornado 是 FriendFeed 使用的可扩展的非阻塞式 web 服务器及其相关工具的开源版本。
主要用来进行进程间接受消息的WebAPI的通信基础。
\item Python-Requests 

这是一个Http请求的第三方库,用于进行进程间http通信。
\item Python-MySQLDB  

python连接数据库的第三方库。
\item Python-SQLAlchemy 

这是资源管理器连接资源监控的数据库的底层。
\item Python-Numpy 

Scipy的底层数学运算的实现,NumPy系统是Python的一种开源的数字扩展。这种工具可用来存储和处理大型矩阵
\item Python-Scipy 

Python的科学计算类库。Python scipy的科学计算应用。包括了数据处理,数值计算等内容,数字信号处理,快速傅立叶变换等应用。
\end{itemize}


\section{实验结果和分析}
本文从资源分配的公平性,资源利用情况和任务实际执行性能这
三个方面对CBDRF进行了测试和仿真。实验模拟生成若干个任务的请求,
分别采用openstack的资源调度器和CBDRF的调度器经行
对比。
\subsection{资源公平性}
首先对分配的公平性进行测试。测试采用随机生成符合正态分布的30个任务对在上述环境的
云计算openstack环境进行虚拟机分配的实验。采用相同的30个任务,分别采用不同的调度
算法进行分析和测试。

首先是,传统的单资源的分配,可以看到,基于Ram进行最大最小原则进行分配的算法确实
在针对内存这一个指标中做到了较优的结果,内存分配在各个物理机之间平均分配。但是,但
并且在任务负重增大的情况下,平台的瓶颈从内存变成CPU时,这种基于单资源的算法无法适
应这种变化。
\begin{figure}[htbp]
\centering\includegraphics[width=0.618\textwidth]{FilterAlloc-crop}
\caption{Filter算法的分配结果}\label{fig:FilterAlloc}
\end{figure}

接下来,是CBDRF算法的分配的结果,如图所示,CBDRF
考虑了作业需求的主导资源,可以看到,CBDRF的分配,整体比较平滑,各个资源间的
差距不大,并且在任务负重增大的情况下,平台的瓶颈从内存变成CPU时,算法可以很好
的适应这种变化。
\begin{figure}[htbp]
\centering\includegraphics[width=0.618\textwidth]{CBDRFalloc-crop}
\caption{CBDRF算法的分配的结果}\label{fig:CBDRFalloc}
\end{figure}


进一步分析,计算各个节点上的domainantShare,CBDRF算法的分配算法的domainantShare较openstack
的filter有非常均衡,而基于内存的单资源的分配上,filter算法明显无法做到相应的结果。
由此得出,如果任务的请求对资源的需求的异质性越大,可以发现原始的基于单资源负载
均衡的调度算法在多资源的情况下分配的很不均衡,基于单资源的排序算法往往会导致其他
资源的负载均衡被破坏。从这里看出,基于DRF的多资源算法可以很好的调和异质任务间资源需求
的差距,同时可以保障分配均有良好的资源情况。
\begin{table}[htp]
\begin{center} 
\caption{分配算法的分配结果}
\begin{tabular}{|c|c|c|c|c|c|}
\hline
\multicolumn{3}{|c|}{CBDRF分配} & \multicolumn{3}{|c|}{Filter分配} \\
\hline
CPU & 内存 & domainant\_share &CPU & 内存 & domainant\_share\\
\hline
14&21&0.072917&13&20&0.067708\\
\hline
13&12&0.067708&11&14&0.057292\\
\hline
13&12&0.067708&15&14&0.078125\\
\hline
15&15&0.078125&12&14&0.0625\\
\hline
15&10&0.078125&7&16&0.036458\\
\hline
15&28&0.078125&16&17&0.083333\\
\hline
15&14&0.078125&23&17&0.119792\\
\hline
15&18&0.078125&18&18&0.09375\\
\hline
\end{tabular}
\end{center}
\end{table}

\begin{table}[htp]
\begin{center} 
\caption{分配算法的Jain 's的系数}
\begin{tabular}{|c|c|c|c|}
\hline
\multicolumn{2}{|c|}{CBDRF分配} & \multicolumn{2}{|c|}{Filter分配} \\
\hline
domain\_jains& 0.996458710066&domain\_jains&0.909810126582\\
\hline 
cpu\_jains& 0.996458710066&cpu\_jains&0.909810126582\\
\hline
mem\_jains& 0.89588634436&mem\_jains&0.984389561976\\
\hline
\end{tabular}
\end{center}
\end{table}

最后利用了Jain's系数计算了各个资源的分配的负载情况,可以看出,基于
CBDRF的分配的各个物理机上的公平指数较openstack的filter\_scheuler有较好的提升,
在CPU和整体的两个角度均要好于只基于内存考虑的filter算法。而
openstack的分配在单个内存的考虑上的负载较为平衡,但其他资源的考虑有所不足。
\subsection{资源利用率}
接下来针对DAG任务的资源利用情况进行模拟和仿真,任务采用多算法的协同求解的PO和MOM算法
进行测试得到的情况如图所示。模拟生成100个不同的任务的基于多算法协同求解的PO和MoM任务进行
计算和统计。进行调度和执行,同时记录集群的实时情况。
\begin{figure}[htbp]
\centering\includegraphics[width=0.618\textwidth]{RequestCPU-crop}
\caption{CPU对比}\label{fig:RequestCPU}
\end{figure}
\begin{figure}[htbp]
\centering\includegraphics[width=0.618\textwidth]{RequestMem-crop}
\caption{内存对比}\label{fig:RequestMem}
\end{figure}

考虑请求资源的优化,上图描述了任务中前20个任务的资源请求解析情况。
这是一个典型的DAG关联的任务。实验表明如果采用基于Merger的合并优化,整个系统的约50\%的作业的请求可以被优化,
若果不考虑DAG作业的相关性,不采取任何优化进行系统的调度,将使得系统整体的CPU和内存
均有所提升,造成大量的空闲资源。

从整个平台可分配的任务数进行统计,由于CBDRF采用了资源复用的优化,使得一个DAG任务
不会一次得到全部的资源请求。而原始的DRF或是filter\_scheduler均未采用上述的优化,使得
在资源的分配过程中大量的资源被空闲计算占用,使得云计算平台的资源的利用率得以提高,使得
整个平台的可以同时运行的任务数大大提升,降低整体的运行时间。另外,由于可运行的任务数的提高
以及资源利用率的提升,使得系统的整体运行性能较原始有所提高。
\begin{figure}[htbp]
\centering\includegraphics[width=0.618\textwidth]{runTimeLoad-crop}
\caption{任务运行统计}\label{fig:runTimeLoad}
\end{figure}

最后,计算测试资源实际利用情况。这里统计实际的空闲情况。
我们可以发现,原始不考虑DAG关系的空闲等待的作业数情况大致
仅为采用合并操作的一倍左右。这是由于大量的后置任务在等待前置任务的结束运行,
而此时这些后置任务却已经分配到资源,产生了大量的空闲等待作业,降低了系统的运行性能。
\begin{figure}[htbp]
\centering\includegraphics[width=0.618\textwidth]{runTimeIdle-crop}
\caption{空闲作业统计}\label{fig:runTimeIdle}
\end{figure}

\subsection{任务执行性能}
最后对任务执行的性能进行测试,这次测试采用通信量较多的FDTD算法的程序进行测试。实验
采用了20个FDTD算法任务进行调度,统计不同调度算法下,任务的执行时间。
\begin{figure}[htbp]
\centering\includegraphics[width=0.618\textwidth]{TaskPerformance-crop}
\caption{运行性能统计}\label{fig:TaskPerformance}
\end{figure}

FDTD算法是典型的通信关联任务,采用通信的基于MPI的多节点程序进行测试。测试进行
20个任务进行调度,计算其运行的平均时间。实验结果表明,基于关联规则合并的CBDRF
较原先的任务将计算的时延优化了15\%左右,如果计算的过程中,通信量的增大会导致
优化进一步加大。原始的调度只考虑负载的运行状态,忽略了任务之间通信的关联性,
导致任务通信的代价成为计算性能的瓶颈。而CBDRF基于关联任务的合并,使得任务之间的数据局部性
得到了良好的保障,提高了整体的计算性能。

\section{本章小结}
本章在由8个工作站构成的计算节点和4台pc机组成的网络和控制节点的openstack平台环境下,
在其平台上针对资源公平性,资源利用率和任务执行性能进行了测试和仿真。
实验表明CBDRF调度策略可以提高平台的利用率和执行性能。

%% ----------------------------------------------------------------------
%%% END OF FILE 
%% ----------------------------------------------------------------------
