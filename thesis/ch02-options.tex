
\chapter{云计算资源管理与调度相关技术}
\label{chap:outline}

对于企业和公司,为了完成各种对外的服务以及公司内部业务逻辑,需要大量
的硬件资源, 而硬件的资源的代价往往比较昂贵,所以如何充分挖掘硬件资源潜力
从而增加硬件的利用率shi yi

\section{OpenStack云计算平台}

OpenStack是一个美国国家航空航天局和Rackspace合作研发的,
以Apache许可证授权,并且是一个自由软件和开放源代码项目。

OpenStack是一个云平台管理的项目,它不是一个软件。这个项目由几个主要的组件组合起来完成一些具体的工作。

OpenStack是一个旨在为公共及私有云的建设与管理提供软件的开源项目。
它的社区拥有超过130家企业及1350位开发者,这些机构与个人都将OpenStack作为基础设施即服务(简称IaaS)资源的通用前端。
OpenStack项目的首要任务是简化云的部署过程并为其带来良好的可扩展性。OpenStack的核心组件有以下9个:
\begin{itemize}
\item 计算(Compute):Nova,Nova是OpenStack云中的计算组织控制器。
支持OpenStack云中实例(instances)生命周期的所有活动都由Nova处理。
其中的调度器nova-schedule作为一个守护进程运行,通过恰当的调度算法从可用资源池获得一个计算服务。
nova-scheduler会根据诸如负载、内存、可用域的物理距离、CPU构架等作出调度决定。
\item 对象存储(Object):Swift,其最初是由 Rackspace 公司开发的高可用分布式对象存储服务,
并于 2010 年贡献给 OpenStack 开源社区作为其最初的核心子项目之一,
为其 Nova 子项目提供虚机镜像存储服务。Swift 支持多租户模式、容器和对象读写操作,适合解决互联网的应用场景下非结构化数据存储问题。
\item 镜像(Image):Glance,用来管理在 OpenStack 集群中的镜像,但不负责实际的存储。
它为从简单文件系统到对象存储系统(如 OpenStack Swift 项目)的多种存储技术提供了一个抽象。除了实际的磁盘镜像之外,它还保存描述镜像的元数据和状态信息。
\item 身份(Identity):Keystone(OpenStack Identity Service)是OpenStack框架中,负责身份验证、服务规则和服务令牌的功能, 它实现了OpenStack的Identity API。
\item 网络地址管理(Network):Neutron是OpenStack核心项目之一,提供云计算环境下的虚拟网络功能.
\item 块存储 (Block):Cinder--提供块存储(Block Storage),类似于 Amazon 的 EBS 块存储服务,OpenStack 中的实例是不能持久化的,需要挂载 volume,在 volume 中实现持久化。Cinder 就是提供对 volume 实际需要的存储块单元的实现管理功能。
\item UI 界面 (Dashboard):Horizon,Horizon套件提供IT人员一个图形化的网页接口,让IT人员可以综观云端服务目前的规模与状态,并且,能够统一存取、部署与管理所有云端服务所使用到的资源。
\item 测量 (Metering):Ceilometer,主要负责监控数据的采集,采集的项目包括虚拟机的性能数据,n
eutron-l3-router使用的网络带宽,glance,cinder,swift等租户使用信息,
甚至是通过snmp采集物理机的信息,以及采集支持opendaylight的网络设备信息。
\item 编配 (Orchestration):Heat 类似于AWS的CloudFormation, heat实现了一种自动化的通过简单定义和配置就能实现的云部署方式。
可以在heat模板中定义连串相关任务,然后交由heat,由heat按照一定的顺序执行heat模板中定义的一连串任务。利用heat还可以连接到neutron来帮助编排负载均衡和其他网络功能。
\end{itemize}
\section{资源管理抽象模型}

对于企业和公司,为了完成各种对外的服务以及公司内部业务逻辑,需要大量

\section{资源管理与调度系统范型}
对于企业和公司,为了完成各种对外的服务以及公司内部业务逻辑,需要大量
\subsection{集中式调度器}
对于企业和公司,为了完成各种对外的服务以及公司内部业务逻辑,需要大量
\subsection{两级调度器}
对于企业和公司,为了完成各种对外的服务以及公司内部业务逻辑,需要大量
\subsection{状态共享调度器}
对于企业和公司,为了完成各种对外的服务以及公司内部业务逻辑,需要大量

\section{资源调度策略}

对于企业和公司,为了完成各种对外的服务以及公司内部业务逻辑,需要大量

\subsection{FIFO调度策略}
对于企业和公司,为了完成各种对外的服务以及公司内部业务逻辑,需要大量
\subsection{公平调度策略}
对于企业和公司,为了完成各种对外的服务以及公司内部业务逻辑,需要大量
\subsection{能力调度器}
对于企业和公司,为了完成各种对外的服务以及公司内部业务逻辑,需要大量
\subsection{延迟调度策略}
对于企业和公司,为了完成各种对外的服务以及公司内部业务逻辑,需要大量

\section{本章小结}

对于企业和公司,为了完成各种对外的服务以及公司内部业务逻辑,需要大量
