
\chapter{虚拟化及资源调度}
\label{chap:outline}
本章介绍了云计算和虚拟化技术产生的背景,发展历程和现状进行了详细的介绍。
同时将目前在使用的虚拟机资源调度系统进行了分析,总结归为四类:平台级调度,
计算框架级调度,主机级调度和进程级调度。

\section{虚拟化技术}

\subsection{虚拟化技术及分类}
虚拟化\cite{ref16}是指计算机元件在虚拟的基础上而不是真实的基础上运行。
虚拟化技术可以扩大硬件的容量,简化软件的重新配置过程。
CPU的虚拟化技术可以单CPU模拟多CPU并行,允许一个平台同时运行多个操作系统,
并且应用程序都可以在相互独立的空间内运行而互不影响,从而显著提高计算机的工作效率。
一般常见的虚拟化应用场景如下所示:
\begin{itemize}
	\item 虚拟机(Virtual machine或VM),可以像真实机器一样运行程序的计算机的软件实现
		\begin{itemize}
			\item 平台虚拟化,将操作系统和硬件平台资源分割开
				\begin{itemize}
					\item 完全虚拟化,敏感指令在操作系统和硬件之间被捕捉处理,客户操作系统无需修改,所有软件都能在虚拟机中运行,
						例如IBM CP/CMS,VirtualBox,VMware Workstation
					\item 硬件辅助虚拟化,利用硬件(主要是CPU)辅助处理敏感指令以实现完全虚拟化的功能,
						客户操作系统无需修改,例如VMware Workstation,Xen,KVM
					\item 部分虚拟化,针对部分应用程序进行虚拟,而不是整个操作系统
					\item 准虚拟化/超虚拟化(paravirtualization),
						为应用程序提供与底层硬件相似但不相同的软件接口,客户操作系统需要进行修改,例如早期的Xen
					\item 操作系统级虚拟化,使操作系统内核支持多用户空间实体,
					例如Parallels Virtuozzo Containers,Unix-like系统上的chroot,Solaris上的Zone
				\end{itemize}
			\item 应用程序虚拟化,在操作系统和应用程序间建立虚拟环境
				\begin{itemize}
					\item 便携式应用程序,允许程序在便携式设备中运行而不用在操作系统中安装
					\item 跨平台虚拟化,允许针对特定CPU或者操作系统的软件不做修改就能运行在其他平台上,例如Wine
					\item 虚拟设备,运行于虚拟化平台之上,面向应用的虚拟机映像
					\item 模拟器
				\end{itemize}
	  \end{itemize}
	\item 虚拟内存,将不相邻的内存区,甚至硬盘空间虚拟成统一连续的内存地址
	\item 存储虚拟化,将实体存储空间(如硬盘)分隔成不同的逻辑存储空间
	\item 网络虚拟化,将不同网络的硬件和软件资源结合成一个虚拟的整体
		\begin{itemize}
			\item 虚拟专用网络(VPN),在大型网络(通常是Internet)中的不同计算机(节点)通过加密连接而组成的虚拟网络,具有类似局域网的功能
			\item 存储器虚拟化,将网络系统中的随机存储器聚合起来,形成统一的虚拟内存池
		\end{itemize}
	\item 桌面虚拟化,在本地计算机显示和操作远程计算机桌面,在远程计算机执行程序和储存信息
	\item 数据库虚拟化
\end{itemize}

\subsection{虚拟化与云计算}
虚拟化技术很早就在计算机体系结构、操作系统、编译器和编程语言等领域得到了广泛应用。
该技术实现了资源的逻辑抽象和统一表示,在服务器、网络及存储管理等方面都有着突出的优势,大大降低了管理复杂度,提高了资源利用率,提高了运营效率,
从而有效地控制了成本。由于在大规模数据中心管理和基于互联网的解决方案交付运营方面有着巨大的价值,
服务器虚拟化技术受到人们的高度重视,人们普遍相信虚拟化将成为未来数据中心的重要组成部分。

虽然虚拟化技术可以有效地简化数据中心管理\cite{ref17}但是仍然不能消除企业为了使用IT系统而进行的数据中心构建、硬件采购、软件安装、系统维护等环节。
早在大型机盛行的20世纪五六十年代,就是采用“租借”的方式对外提供服务的。
IBM公司当时的首席执行官Thomas Watson曾预言道:“全世界只需要五台计算机”,
过去三十年的PC大繁荣似乎正在推翻这个论断,人们常常引用这个例子,来说明信息产业的不可预测性。
然而,信息技术变革并不总是直线前进,而是螺旋式上升的,半导体、
互联网和虚拟化技术的飞速发展使得业界不得不重新思考这一构想,
这些支撑技术的成熟让我们有可能把全世界的数据中心进行适度的集中,
从而实现规模化效应,人们只需远程租用这些共享资源而不需要购置和维护。

云计算\cite{ref26}是这种构想的代名词,它采用创新的计算模式使用户通过互联网随时获得近乎无限的计算能力和丰富多样的信息服务,
它创新的商业模式使用户对计算和服务可以取用自由、按量付费。
目前的云计算融合了以虚拟化、服务管理自动化和标准化为代表的大量革新技术。
云计算借助虚拟化技术的伸缩性和灵活性,提高了资源利用率,
简化了资源和服务的管理和维护;利用信息服务自动化技术,
将资源封装为服务交付给用户,减少了数据中心的运营成本;利用标准化,方便了服务的开发和交付,缩短了客户服务的上线时间。
云计算的发展历程大致如下:
1983年,太阳电脑(Sun Microsystems)提出“网络是电脑”(“The Network is the computer”)。

2006年3月,亚马逊(Amazon)推出弹性计算云(Elastic Compute Cloud;EC2)服务。

2006年8月9日,Google行政总裁埃里克·施密特(Eric Schmidt)在搜索引擎大会(SES San Jose 2006)
首次提出“云计算”(Cloud Computing)的概念。
Google“云端计算”源于Google工程师克里斯托弗·比希利亚所做的“Google 101”项目。

2007年10月,Google与IBM开始在美国大学校园,包括卡内基梅隆大学、麻省理工学院、斯坦福大学、加州大学柏克利分校及马里兰大学等,
推广云计算的计划,这项计划希望能降低分散式计算技术在学术研究方面的成本,
并为这些大学提供相关的软硬件设备及技术支援
(包括数百台个人电脑及BladeCenter与System x服务器,这些计算平台将提供1600个处理器,支援包括Linux、Xen、Hadoop等开放源代码平台)。
而学生则可以透过网络开发各项以大规模计算为基础的研究计划。

2008年1月30日,Google宣布在台湾启动“云计算学术计划”,将与台湾台大、交大等学校合作,将这种先进的大规模、快速计算技术推广到校园。

2008年7月29日,雅虎、惠普和英特尔宣布一项涵盖美国、德国和新加坡的联合研究计划,推出云计算研究测试床,推进云计算。
该计划要与合作伙伴建立6个数据中心作为研究试验平台,每个数据中心配置1400个至4000个处理器。
这些合作伙伴包括新加坡资讯通信发展管理局、德国卡尔斯鲁厄大学Steinbuch计算中心、美国伊利诺伊大学香宾分校、英特尔研究院、惠普实验室和雅虎。

2008年8月3日,美国专利商标局网站信息显示,戴尔正在申请“云计算”(Cloud Computing)商标,
此举旨在加强对这一未来可能重塑技术架构的术语的控制权。
戴尔在申请文件中称,云计算是“在数据中心和巨型规模的计算环境中,为他人提供计算机硬件定制制造”。

2010年3月5日,Novell与云安全联盟(CSA)共同宣布一项供应商中立计划,名为“可信任云计算计划(Trusted Cloud Initiative)”。

2010年7月,美国国家航空航天局和包括Rackspace、AMD、Intel、戴尔等支援厂商共同宣布“OpenStack”开放源码计划,
微软在2010年10月表示支持OpenStack与Windows Server 2008 R2的整合;
而Ubuntu已把OpenStack加至11.04版本中。
2011年2月,思科系统正式加入OpenStack,重点研制OpenStack的网络服务。

随着云计算的越来越被人们广泛使用,虚拟化手段显得尤为重要。
而在众多虚拟化的技术中,虚拟化的多点协作,如何有效的分配和利用资源显得尤为重要。

\section{资源调度分类}
目前的虚拟机资源系统从上往下一般分为平台级调度,计算框架级调度,主机级调度和进程级调度。
\subsection{平台级调度}
由于云计算的应用越来越广泛,各个公司都开始购买各IT公司的云计算服务。
各个IT公司也提供了了不同的云计算服务平台供各个公司学校使用。
比如微软的Windows Azure平台,美国宇航局的OpenStack的平台,亚马逊的EC2平台以及
Google的GAE平台。由于公司的实际业务需求,有时会购买多个服务商的服务\cite{ref1,ref3}。
但每个公司或是学校的需求实际上在各个时间段是不尽相同的。
考虑到实际花费问题等等,公司或学校希望可以在切换各个平台上的任务使得自己的花费尽可能小。

在这些调度其中,用户根据需要决定使用哪个平台。
然后再由实际平台决定资源的分配。
也就是说系统是个两层调度,第一层管理和调度使用平台,第二层为平台自身调度资源。

在实际应用中,任务间是互相隔离的。
保证任务互相不会互相影响,但是往往在单个集群上利用率较低。
但是用户的维护费用和花费较小,属于是面向用户的一种类似经济人的代理部署调度系统。
\subsection{计算框架级调度}
随着互联网的高速发展,基于数据密集型应用的新型计算框架层出不穷。
期间经历了支持离线处理框架MapReduce到支持在线处理框架Storm的转变,
以及后来的出现的迭代式计算框架Spark和流式计算框架S4等\cite{ref15}。
它们基于各自的应用背景,解决了某一类特定的具体问题。
目前在商业公司中,由于各个业务的实际应用背景不同,一般采取的计算框架也不尽相同。
考虑到资源利用率,运维成本,数据共享等因素,公司一般希望将所有这些框架部署到一个公共的集群中,让它们共享集群的资源,并对资源进行统一使用。
这样,便诞生了资源统一管理与调度平台,典型代表是Mesos和YARN。

在这些调度平台中,所有接入的框架要先向该全局资源管理器申请资源,申请成功之后,
再由框架自身的调度器决定资源交由哪个任务使用,
也就是说,整个大的系统是个双层调度器,第一层是统一管理和调度平台提供的,另外一层是框架自身的调度器。

相比每个计算框架分配一个集群的调度而言,
由于作业自身的特点或者作业提交频率等原因,集群利用率较低。
当将各种框架部署到同一个大的集群中,进行统一管理和调度后,由于各种作业交错且作业提交频率大幅度升高,
则为资源利用率的提升增加了机会。
\subsection{主机级调度}
这一层次的调度是各个目前云计算平台的中调度的核心调度部分。
这个调度器的设计的好坏决定这云计算平台实际性能的优劣。目前各个平台,计算框架
根据自己的实际应用情况采用了不同的调度算法,如Hadoop中使用的是fair scheduler,
在OpenStack中使用nova-scheduler确定在哪一个计算节点上创建新的虚拟机等等。

在这一层次调度器中,调度器根据任务的情况和当前所有计算节点的情况决定是将任务分发到具体的某个计算节点中去。
而实际的执行过程中则有具体的虚拟机调度决定哪个任务或是进程应当执行。

与平台级和计算框架级调度相比的话,这一层次的调度器就是比较偏向底层的调度器。
这种调度器是每个云计算平台的一个核心组件,一个高效的调度算法是设计这一层次调度器的核心考虑因素之一。


\subsection{进程级调度}
由于目前绝大多数的云计算平台是基于虚拟化技术实现的,即在一个宿主机上部署着多个虚拟计算节点。
而任务分配到各个虚拟节点后,实际的任务调度就由这个虚拟机的调度器进行资源的分配和调度。
在Xen中有着自己的调度器比如基于Credit算法的调度器,基于SEDF的调度器和早先基于BVT算法的调度器。
而在KVM中则直接使用宿主机Linux系统的调度器。具体的应用条件不同决定了调度器的设计思想和思路不同。

与主机级调度相比,这一调度器更象是操作系统的资源调度器。所以很多这一层次的调度器的设计采用了操作系统中
资源分配的LRU,FIFO等设计思想。但是由于云计算平台是一个多机协作计算的平台,单一的只考虑本机的情况,
或是不将本机的资源分配通知情况上层调度器将会造成实际的调度上下脱节,影响实际平台的调度情况。











\section{本章小结}
 本章详细介绍了云计算和虚拟化技术的背景及其意义并且详细的介绍虚拟化技术常见形式和特点。
 在此基础之上,将虚拟机资源调度按照应用背景和层次进行了分类总结,详细的分析了平台级,计算框架级,
 主机级和进程级调度等常见调度形式。
