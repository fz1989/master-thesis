%% ----------------------------------------------------------------------
%% START OF FILE
%% ----------------------------------------------------------------------
%% 
%% Filename: ch06-conclusions.tex
%% Author: Fred Qi
%% Created: 2012-12-26 13:11:10(+0800)
%% 
%% ----------------------------------------------------------------------
%%% CHANGE LOG
%% ----------------------------------------------------------------------
%% Last-Updated: 2012-12-26 13:11:25(+0800) [by Fred Qi]
%%     Update #: 1
%% ----------------------------------------------------------------------


\chapter{总结与展望}
\label{cha:conclusions}
\section{总结}
云计算是目前互联网时代的重要产物,在云计算平台上。各个公司和研究机构可以获得大量的资源,而云计算平台上会运行大量的
计算任务,而如何协调云计算平台任务的调度器一直是云计算平台设计的核心问题。本文在云计算协同计算的背景下,介绍了现有
的云计算调度器的设计和调度模型。同时结合图论将任务关联性和资源调度加以结合,
将数据的局部性加以考虑,在负载均衡和资源分配公平的基础上,设计和实现了CBDRF的调度算法。结合全文,本文的工作和内容
主要体现在以下的几个方面:
\begin{enumerate}
\item 在云计算协同计算的背景下,对现有的云计算的调度器进行了详细的分析,总结了从全局单一调度器,两层调度器和状态共享
调度器等调度分配模型。
\item 同时,对云计算的任务结合图论,对任务的调度模型采用图论的方式进行解释。给出了DAG任务,通行关联任务和无关联任务
的定义。
\item 结合DRF算法,结合数据局部性理论和图论相关理论,提出了基于任务之间关联关系的CBDRF算法,用以提高云计算平台的资源
利用率和执行作业性能。同时给出了该算法的伪代码和计算流程。
\item 在CBDRF的基础之上,给出了基于openstack的CBDRF调度器的系统设计和实现,详细的分析了CBDRF调度器的资源管理模块,
任务解析器和资源调度迭代去。
\item 最后,与原有的openstack算法和DRF等算法进行了仿真测试,进行了对比实验。实验数据表明CBDRF调度算法能够提高整个平台
的资源的利用率和整体云计算平台的计算性能
\end{enumerate}

\section{进一步工作}
结合上述工作,本文进一步的工作主要集中在以下几点:
\begin{enumerate}
\item CBDRF主要进行资源的分配和调度,在调度的过程中考虑任务数据的局部性和平台的负载,但是对于优先级的任务以及任务之间的
抢占等情况的处理仍未加以考虑,这是CBDRF调度器下一步要支持的工作。
\item 另外,CBDRF调度的调度策略仍然是一个轻量的中央调度器。由于中央调度器的在并发任务的处理上时云计算平台的提升性能的主要瓶颈。
因此优化CBDRF调度器的并发操作仍是进一步工作的重点。
\item 最后CBDRF的调度器与传统的调度器一样。这个调度器没有考虑动态的资源迁移问题。这个调度器的实现依然采用了静态分配资源的
策路,支持动态去中心的调度是下一步工作的。
\end{enumerate}

%% ----------------------------------------------------------------------
%%% END OF FILE 
%% ----------------------------------------------------------------------
